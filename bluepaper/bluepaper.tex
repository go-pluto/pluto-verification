\documentclass[a4paper]{scrartcl}

\usepackage[english]{babel}

\usepackage[sc,osf]{mathpazo}

\setkomafont{section}{\normalfont\bfseries\Large}
\setkomafont{subsection}{\normalfont\bfseries\large}
\setkomafont{subsubsection}{\normalfont\bfseries\normalsize}
\setkomafont{paragraph}{\normalfont\bfseries\normalsize}
\setkomafont{title}{\normalfont}

\usepackage[utf8]{inputenc}
\usepackage[T1]{fontenc}

\usepackage{amsmath}
\usepackage{amssymb}
\usepackage{amsthm}
\usepackage{stmaryrd}
\usepackage{url}
\usepackage{color}
\usepackage{pagecolor}

\usepackage[style=ieee, backend=biber]{biblatex}
\addbibresource{bibliography.bib}

\definecolor{background}{HTML}{B3E5FC}

\pagecolor{background}

\usepackage[bottom=4.5cm,inner=4cm,outer=3cm,marginparwidth = 2.1cm,
headsep = 1.5cm
]{geometry}

\usepackage{hyperref}
\hypersetup{
     colorlinks   = false,
     pdfauthor = the pluto devs,
     pdftitle = pluto and the IMAP CRDT
}

%%%%%%%%%%%%%%%%%%%%%%%%%%%%%
% for own macros

%%%%%%%%%%%%%%%%%%%%%%%%%%%%%%%

\title{pluto and the IMAP CRDT}
\author{- work in progress -}

\begin{document}

\date{}
\maketitle

%\begin{abstract}
% TODO: add abstract
%\end{abstract}

\section{Motivation and Goals}

\section{System Model}\label{sec:sysmodel}
We base our system model on the one that was initially introduced with
the CRDTs by Shapiro et al \cite{shapiro_crdt}:

\begin{itemize}
  \item A finite set of processes $\mathcal{P}$. One process can be understood
  as a pluto worker instance.
  \item An asynchronous network of communication channels.
  \item Processes can send messages to other processes. We will later define the
  set of allowed messages.
  \item Processes change their internal state as a result of a received message
  \emph{or} because of user input.
\end{itemize}

Within our model, we define the following set of failures:

\begin{itemize}
  \item A process can crash. A crashed process can recover with intact memory,
  meaning that the state of a process is either lost forever, or
  recovers completely.
  \item The network can partition and recover. Messages in transit can be
  delayed for an arbitrary time.
\end{itemize}

As user input, we allow the following set of IMAP commands, which will be later
defined as operations on a CRDT:

\[\texttt{CREATE, APPEND, DELETE, STORE}\]

The messages in the network are downstream updates only. Each IMAP command on
one process immediately triggers an update of the state if the process and
downstream update messages to all other processes.

\section{IMAP CRDT}

We present an IMAP CRDT that features the presented IMAP commands. At this point,
we should probably start to describe our payload, since this is the most critical
part of our CRDT.

\begin{itemize}
  \item \textbf{Payload:} We start by illustrating the payload without providing
  the precise definition of the types. Basically, the payload consists of two
  sets:

  $\{(\text{uni},\texttt{\#1}), (\text{inbox},\texttt{\#5}), (\text{uni},\texttt{\#2})\}$

  $\{(\text{uni},(\llparenthesis \text{yolo}, \texttt{u4}, \{\text{seen}\} \rrparenthesis,\texttt{\#5}),
  (\text{inbox},(\llparenthesis \text{swag}, \texttt{u17}, \{\text{recent}\} \rrparenthesis,\texttt{\#2})\}$

  The first set is an OR-set of folder names and unique tags, identified by a $\#$hashtag.
  The second set is a relation between foldernames and OR-sets with messages and uniqe tags as payload.
  This set is designed to be a one-to-one relation. A message is a record of the message content, a message id and
  a set of flags.

  \item \textbf{CREATE:} Essentially, CREATE adds an item in the first set. If the second set contains
  no relation for the foldername, a relation between the foldername and an empty OR-Set is created.
  The new pair or foldername and unique tag is pushed downstream.
  \item \textbf{DELETE:}
  \item \textbf{APPEND:}
  \item \textbf{STORE:}

\end{itemize}


\printbibliography


\end{document}
